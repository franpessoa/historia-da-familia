\chapter{}
De Conceição do Araguaia, ficaram alguns outros casos dignos de lembrança, dois deles exemplos de mais alguns milagres operados pela medicina dos nossos sertões: 
Um dia, Paulo me ligou da Nazaré, avisando que providenciasse uma internação para o Daio, um esperto peão da fazenda que se acidentara no pasto.
Galopando na disparada atrás de uma rês, o burro que ele montava enfiara a pata num buraco, e montaria e cavaleiro foram de ponta-cabeça ao chão.
O burro morreu na hora.
Daio sobreviveu em semi-coma e parecendo um extraterrestre: um corpo franzino encimado pelo crânio dilatado por um imenso edema.
No hospital, sugeriram-me chamar um neurologista que, por sorte, um concurso do Banco do Brasil desterrara para Conceição do Araguaia.
Convocado, em poucos minutos o médico surgiu à porta e após examinar o peão, pediu uma radiografia.
A enfermeira arrastou até o quarto um aparelho avoengo que, entre tremeliques e rangidos, emitiu o que parecia, após a revelação, o retrato de uma nebulosa.
Desconsolado, o jovem egresso da Faculdade de Medicina da Universidade de São Paulo, com especialização em cirurgia neurológica, ergueu o exame contra a janela e gemeu:
``--Minha senhora, isso aqui pode ser qualquer coisa, um cérebro, um pulmão, um estômago ou um intestino.
Ou seja, vamos caminhar no escuro.
Que Deus nos proteja, a mim e a ele!'' 
Daio vegetou por semanas naquela cama, sem dar acordo de nada.
Um dia, despertou, levantou-se e foi embora.
Para Brasília.
Onde se tornou um vendedor de sucesso.

De outra vez, o gerente da Fazenda Curral de Pedra acordou-nos, noite alta, para comunicar que estava remetendo rio abaixo uma voadeira, trazendo a bordo um outro peão, semi-morto.
O infeliz, após o jantar, envolvera-se numa briga durante a qual lhe assestaram no cocuruto um golpe de machete tão vigoroso que uma banda da cara caiu para o lado, pendurada apenas pela pele do pescoço.
O talho resvalara pelo canto da sobrancelha, levando a orelha, fazendo saltar uns estilhaços de ossos, deixando um pouco de miolos à mostra e se detivera no maxilar inferior, após decepar um pedaço de bigode e espalhar alguns dentes pelo chão.
Apavorados, os colegas trataram de deitá-lo no fundo do barco, juntaram o naco da cabeça ao restante e aproveitando que o coitado desmaiara, puseram-lhe a mão inerte sobre o arranjo para que não se desmantelasse na viagem.
Em vão.
O pândego do médico que foi esperá-lo na beira do rio, Dr. Pedro Monteiro, acho que era esse o nome, contou que as duas horas e tanto de sacolejo nas águas  do Araguaia anarquizaram tudo e a cara do peão chegou parecendo uma tela de Picasso.
Tudo bem fixado pelo sangue que secara.
Todavia, o sujeito ainda respirava.
Como era madrugada e não havia energia na cidade, a vizinhança foi acordada para trazer seus carros para perto da janela do centro cirúrgico e acender os faróis altos.
Foi assim que o médico pode dar inicio à tentativa de salvamento.

Muitos meses depois, tarde da noite, uma úlcera péptica que me atormentou durante alguns anos, resolveu dar o ar da graça.
Paulo estava fora da cidade e, contorcendo-me de dor, não tive outra saída a não ser telefonar para o Doutor Pedro, aquele mesmo, o pândego que operara o tal peão.
No escuro, tateando pelas paredes, tropeçando nos buracos do calçamento, ele chegou fazendo piada como sempre, maldizendo os maridos que abandonam as esposas carentes para dar trabalho aos médicos e se mandam para a farra sabe Deus onde!  E enquanto me aplicava a injeção de sedativo, contou:

\textit{``-- Você não imagina o que me aconteceu, no caminho para cá.
Eu vinha nessa escuridão, andando rente aos muros, quando de um vão de porta saiu uma criatura me chamando pelo nome e perguntando se eu tinha fogo.
Assustado, saquei o isqueiro e, à luz da chama, vi a cicatriz que vinha do alto da cabeça até o queixo.
Era ele, torto e feio, mas vivo! O peão do Curral de Pedra, acredita? Acendeu o cigarro, perguntou pela minha saúde, agradeceu e foi-se embora, no rumo do rio.
O homem está vivo e, pelo menos, a memória parece que está funcionando! Veja você!'' }

Vivia, na Fazenda Nazaré, num barraco erguido no limite mais distante da propriedade, um jovem casal de retireiros, Antonio e Maria.
Arredios, cabeludos, maltrapilhos e sujíssimos, eram a encarnação do elo perdido.
Isso até o Paulo assumir a administração da fazenda e resolver civilizá-los a muque.
Maria apenas grunhia e Antonio usava com alguma dificuldade um vocabulário bastante restrito.
Daí porque nunca conhecemos de fato a história dos dois.
Acho que eram andarilhos que em algum tempo e em algum lugar se juntaram e vieram encalhar naquele canto do mundo.
Apenas uma coisa ficava bem clara em relação aos dois: amavam-se com um amor absoluto, indissolúvel, indissociável, na vida ou na morte.


Paulo deu-lhes a atribuição de vigiar as cercas, como competia à função de retireiros, e avisar sobre qualquer tentativa de invasão das terras.
E, para tanto, conferiu um salário ao Antônio.
Em troca, exigiu que cortassem o cabelo, se vestissem e se banhassem como gente.
E que dessem um jeito naquela toca, dando-lhe o aspecto de casa.
Foi a partir daí que eles começaram a aparecer lá pela sede e pudemos divisar-lhes finalmente as feições.
Antonio até tinha traços bonitinhos, mas Maria, coitada, feia, magra e olhuda, não tinha remédio.
Ainda por cima, cheirava como onça.

Um dia, logo cedo, apareceram-me os dois na porta da cozinha.
Naquele seu jeito estropiado e agoniado de falar, Antônio pediu que eu desse um jeito de a mulher dar cria, pois que ela perdia todos os meninos, nenhum vingava.
Com cautela, expliquei a ele que teríamos que levar Maria a um médico para descobrir a causa do problema.
Desconfiado, não parecia disposto a concordar.
Queria só um remédio para ela tomar.
Não queria homem nenhum mexendo na sua Maria.
Ao fim de algum tempo de argumentação, e jurando que não sairia do lado dela nem por um segundo, consegui que ele deixasse a companheira ir comigo ao posto de saúde.

Assim que Antonio saiu portão afora, preparei-me para a batalha.
Primeiro, um banho, anunciei a Maria, levando-a até o banheiro, mostrando-lhe o chuveiro, abrindo-o e explicando como era tomar banho.
Deixei-a por um instante e fui buscar uma toalha.
Quando voltei, encontrei-a ainda vestida, com a matula debaixo do braço, encharcando-se dos cabelos aos chinelos sob o jato morno do chuveiro aberto ao máximo.
Rindo, feliz.
Saí para comprar outras roupas enquanto as empregadas tratavam de desencardi-la com sabão e bucha.
A luta para manter a criatura limpa e desinfetada durou meses, mas, para alegria do Antonio, Maria pariu.
Quando nos mudamos de Conceição do Araguaia, o moleque sobrevivia, galhardamente.

Num fim de tarde, sol se pondo, Paulo voltava da fazenda com o Fernando, quando notaram intenso e cambiante clarão que ziguezagueava pelo céu com velocidade espantosa.
A intrigante aparição acompanhou-os por algum tempo e desapareceu.
Ainda sem querer acreditar, chegaram em casa anunciando que talvez tivessem visto um disco voador.
No dia seguinte, só se falava disso na cidade.
Muitos outros tinham presenciado o mesmo fenômeno.
Mas, a ninguém ocorreu rechaçar por quaisquer meios a suposta incursão alienígena.
Só ao valente Antônio e pelas razões que relato a seguir:
O retireiro materializou-se logo cedinho, na porta da cozinha.
A indignação era tanta que as palavras se embaralhavam mais do que de costume em sua boca trêmula.

\textit{``-- Calma, homem, fale devagar''}, pedimos.

\textit{``O que aconteceu?''}

\textit{``-- Uma coisa que veio do céu querendo levar minha Maria!''}, exclamou ele.

\textit{``-- Que coisa, Antônio, como era isso?''}

\textit{``-- Uma coisa grande, redonda e bem brilhosa que veio para cima da minha Maria, ficando assim bem em cima dela.
Andava quando ela andava e parava quando ela parava.
Maria começou a gritar e aí mesmo é que a coisa foi chegando mais perto.''}

\textit{``-- E então, Antonio, o que é que você fez?''}

Ele, triunfante:
\textit{``-- Fui lá dentro, peguei minha pica-pau e acertei uns bons tiros neles.
Aí eles foram subindo, subindo e saíram zunindo p’ras bandas de cá.
Não voltaram mais!''}

Foi assim que Antonio, sozinho, abortou quiçá a primeira tentativa de abordagem de um terráqueo de Conceição do Araguaia por seres de outro planeta.
Ficou também esclarecido, sem dúvida, o porquê daquele ziguezaguear desnorteado e da velocidade vertiginosa imprimida ao estranho artefato.
Só ficou sem resolver, em virtude da apressada partida, o fascinante enigma de, por que diabos, eles se interessaram justamente pela Maria!

Iracema e Deusa eram duas primas que vieram trabalhar em casa como domésticas, poucos meses depois que nos estabelecemos em Conceição.
Ficaram conosco pelo tempo em que permanecemos na cidade e ainda nos acompanharam até Teresina, morando conosco por mais um ano.
Iracema tornou-se exímia cozinheira.
Deusa cuidava da limpeza e passava roupa.
Entreolharam-se espantadas quando, ao contratá-las, anunciei que teriam registro em carteira e receberiam o salário mínimo.
Nessas regiões distantes dos grandes centros, a duas décadas do final do século XX, as empregadas domésticas ainda trabalhavam por pouco mais que a comida.

Iracema era alegre e ostentava com orgulho o título de rainha do forró de Conceição do Araguaia.
Deusa era séria, desconfiada e a custo aprendeu a sorrir.
Chegavam por volta de oito da manhã e iam embora às duas horas da tarde.
Como à tarde eu estudava com os meninos, ao fim de algum tempo, satisfeita como estava com o desempenho das primas, ofereci pagar mais meio salário a cada uma, para que ficassem por mais duas horas para me ajudar.
Depois de revirar os olhinhos em atitude de séria reflexão, Iracema respondeu pelas duas:

\textit{``-- Não, Dona Teresa, não vai dar certo, não.
De repente eu fico cansada e vou até perder a vontade de ir para o meu forró.
Minha vida está tão boa do jeito que está.
Para que vou mudar isso? Só por causa de dinheiro? Não, muito obrigada, vamos deixar do jeito que está mesmo.''}

Deusa apenas anuiu com a cabeça, dando total apoio ao arrazoado da prima.

Não estranhei a atitude das duas, porque meses antes, eu tivera uma amostra do jeito paraense de ser.
Íamos chegando de carro a Belém, quando pouco antes da entrada da cidade, vi um caboclo descalço e maltrapilho saindo do mato com uma vara aonde ia espetada uma meia dúzia de maravilhosos cajus.
Calor, sede matando, pedi ao Paulo que parasse e comprasse os cajus, pelo amor de Deus.
Abordado, o dono das frutas gentilmente declarou que os cajus eram dele, não eram para vender, não.
Paulo insistiu: 

\textit{``--Sei que você os apanhou de graça, no mato, e sei que se voltar lá, tem mais.
Por que não me vende esses? Minha mulher está morrendo de sede.
Faça o preço, eu pago.''}

\textit{``--Moço, se sua mulher está com sede, eu até dou uns dois para ela, leve.
Mas, vender meus cajus não vendo, não.
Esses, eu apanhei para mim.''}

E lá se foi, estrada afora, altivo como um príncipe.

Muito tempo depois, quando nos mudamos para Teresina, nossa casa tinha um apartamento nos fundos, onde alojei Iracema e Deusa, da melhor maneira que pude, agradecida por terem se disposto a ir conosco.
Matriculei-as numa escola próxima de casa, pois estudar numa escola regular era o maior sonho das duas.
Pareciam muito satisfeitas.
Pouco tempo depois, Iracema quis saber onde se dançava forró em Teresina.

\textit{``-- Em todo lugar, Iracema''}, respondi; \textit{``os teresinenses se gabam de dançar o melhor forró de todo o Nordeste.''}
Realmente, forró era assunto sério para aquela gente e dançado num ritmo tão miudinho, endiabrado e acelerado que mal começado o baile, o suor já descia em bagos pela testa, pelas costas e pelo peito dos pares.
Eles, porém, testa franzida, expressão concentrada, não paravam e naquele pula-pula e rala-rala, iam até o dia amanhecer.

Iracema se sentiu picada, deu para perceber, e saiu a campo, decidida a impor sua realeza nas pistas de dança da cidade.
Diga-se a seu favor que insistiu por meses, sem esmorecer.
Ao final de um ano, rendeu-se, derrotada:

\textit{``-- O diabo que entenda o jeito dessa gente dançar forró.
Eu vou é embora para minha terra que lá, sim, o povo me dá valor.''}

E foi mesmo, não teve quem a segurasse.
Levando a reboque a prima Deusa, sua escudeira fiel.

Por fim, uma singela lição do Antônio da Rosa, o esperto capataz da Fazenda Maria Luísa, dos irmãos Gomes dos Reis:

\textit{``-- Patrão é uma coisa fácil de se lidar -- é só dizer o que eles querem ouvir.''} 

Nunca esqueci.
