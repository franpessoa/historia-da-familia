\chapter{}
Quando morávamos junto da Fábrica do meu pai, no quarteirão defronte, num canto do terreno ocupado pelo almoxarifado da Prefeitura, instalavam-se os parques de diversão e circos mambembes que passavam pela cidade.
Num tempo em que ainda não se ouvia falar de \textit{playcenters}, \textit{playgrounds} e \textit{videogames}, a chegada de um deles punha a meninada em completo alvoroço.
Se era um circo, acompanhávamos dia após dia o descarregar dos caminhões, ansiosos para saber se era do tipo que trazia tigres, leões e elefantes, ou, mais empolgante ainda, o globo da morte com seus intrépidos motociclistas e suas máquinas barulhentas.
Na maior parte das vezes, porém, o que aparecia eram os chamados circos-teatro, quase indigentes, em que o ralo elenco, dono incluído, revezava-se nos malabares, no trapézio, na bilheteria e, ao grande final, subia ao palco para encenar o ponto alto do espetáculo, o drama em três atos sempre farto em suspense, lágrimas e punhaladas que encerrava a noite.
Mais divertido para nós, era descobrir, sob a pesada maquilagem, o palhaço agora travestido de pai cruel da infeliz mocinha em quem reconhecíamos a bailarina da corda bamba, chorando copiosamente de amor contrariado pelo galã, que pouco antes víramos a dar saltos mortais, pendurado às barras do trapézio.

Nesses circos-teatro em que o magro programa arrastava-se em perfeita consonância com a penúria do empresário e dos artistas, organizavam-se, nos intervalos entre um número e outro, concursos de cantores, de lutadores, de dançarinos, recurso infalível para acender a imaginação dos candidatos a celebridade e salvar a bilheteria do dia.

Uma vez, tinha eu sete ou oito anos, sabe-se lá por que, em absoluto segredo, inscrevi-me para cantar num desses concursos.
Eu fazia parte do orfeão da escola e era elogiada pela minha afinação.
O concurso seria na matinê de domingo.
Aproveitando que meus pais sempre tiravam uma soneca após o almoço desse dia, saí despercebida de casa, encarei a batalha de vencer a timidez e a pouca confiança na minha aparência mais do que rechonchuda e me apresentei ao organizador do concurso, na coxia.
Ao seu sinal, subi ao palco e assim que anunciado meu nome, comecei a cantar um bolero, imaginem só.
Foi aí que lá de cima, olhando aquele monte de desconhecidos me fitando, dei-me conta do ridículo e do descabido da situação.
Vacilei, e pior, nesse momento o palhaço aproximou do meu rosto o seu carão brilhante de suor e alvaiade e, no susto, minha voz foi sumindo num fio e desapareceu por completo.
E eu nem sabia que nessa altura meus pais, alertados pelo som do meu nome no alto-falante, tinham acordado e, estupefatos, procuravam entender como sua tímida e roliça pata-choca decidira tentar o estrelato.
Saí debaixo de vaias e aplausos de consolação.
Se bem me lembro, papai e mamãe tiveram a gentileza de não aprofundar o assunto diante do meu furioso silêncio.
Já o mano e seus amiguinhos da rua não deixaram passar a oportunidade, fazendo do meu vexame a manchete da semana.
Aguentei calada.
Achei que merecia.
Contudo, eu logo teria uma meia vingança.

Quando pouco depois nos mudamos para a casa nova, do outro lado da cidade, coincidentemente estendia-se diante dela um imenso terreno baldio, também de propriedade da Prefeitura e para o qual, a partir dessa época, seria transferida a montagem de parques e circos de passagem pela cidade.
Um dia, um vizinho, rindo, veio chamar meu pai para espiar seu promissor rebento, Reginaldo, de tabuleiro pendurado ao pescoço, vendendo amendoim torrado nas arquibancadas.
Meu irmão tinha arrumado o negócio com o pipoqueiro, em troca de entrar todo dia no circo, de graça.
Por pouco, não tomou uma coça.
Foi poupado pelo espírito de iniciativa que revelou, o que, no meu caso, ninguém levou em consideração.
Bem, pelo menos não tomei pito como ele.

O que eu gostava mesmo era dos parques de diversão com seus carrinhos coloridos, a roda gigante, as barracas de tiro ao alvo, de jogo de argolas, de pescaria, que me faziam sonhar com a boneca de vestido rodado ou o urso de pelúcia oferecidos como prêmio.

Eu não me arriscava nos brinquedos que giravam vertiginosamente provocando náusea, ou ameaçavam precipitar a gente no vazio.
Nunca compreendi onde está a graça de associar diversão à tortura.
O brinquedo de que mais gostava eram uns barquinhos, dentro dos quais, puxando uma corda passada em torno de uma roldana suspensa, a gente conseguia balançar cada vez mais forte, cada vez mais alto, dependendo só da força dos nossos braços.
Pelo preço do ingresso, o dono do brinquedo nos deixava ficar lá por uns quinze minutos, o que sempre me parecia pouco demais.
Bom mesmo era quando o sujeito se engraçava por uma daquelas mocinhas que flanavam por ali e, empenhado na conquista, esquecia-se do tempo.
Os minutos de lambuja que ganhávamos dessa maneira pareciam os melhores.
Um olho na corda, outro no namoro, eu tratava de ir cada vez mais alto e o medo de que o homem acordasse para o prejuízo e empurrasse para baixo do meu barco aquele calço de borracha que decretava o fim da brincadeira, chegava a dar um frio na barriga, de tanta tensão.

Muitas vezes, quando penso na Morte, o que me vem inevitavelmente à lembrança, como irresistível analogia, é aquele momento de agônica ansiedade durante o qual eu espreitava o tal sujeito, temendo ver seu olhar indiferente voltar-se na minha direção e, sem nem se importar com minha expressão de súplica, meter por baixo do meu barquinho aquele naco de pneu que, por mais que eu puxasse a corda com todas as minhas forças, ia inexoravelmente segurando-o, segurando-o, até detê-lo por completo.
Ah, se ela se distraísse por um instante e me deixasse brincar só um pouquinho mais\dots