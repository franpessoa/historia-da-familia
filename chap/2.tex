\chapter{}
Para o bem e para o mal, minha infância transcorreu numa pequena galáxia de clãs interligados, de origem imigrante e pobre, dos quais os mais importantes eram os Filpi, os Credidio e os Ópice, todos com raízes na Itália. 
Dezenas de olhos e bocas para censurar e orientar e montes de braços para acarinhar e acolher, e uma intensa experiência humana para vivenciar, recolher, degustar e aprender. 

Nossa antiga sala de jantar exibia uma mesa particularmente desgraciosa, quadradona e pesada, mas que, para mim, tinha um encanto especial: embutia um pequeno armário no seu único e largo pé central, onde eram guardados os jornais velhos. 
Um dia descobri que eu cabia direitinho naquele esconderijo e ali, insuspeitada, colhi minhas primeiras e fortes impressões do mundo adulto.  
Era à volta daquela mesa que as visitas se reuniam para o café. 
Era divertido acompanhar, debaixo dela, o movimento das pernas inchadas e nodosas de varizes das tias-avós, na excitação de comentar os acontecimentos da cidade e a vida alheia. 
Quando mamãe e suas irmãs aparteavam a discussão, vez ou outra, contrapondo argumentos mais modernos, arejados por visões do pós-guerra de liberação feminina, difundidas pelo cinema norte-americano quase sempre, as vozes se alteravam, os pés se agitavam e, na ânsia de vencer a contenda, de algum ponto da mesa vinha a cartada definitiva, o disse-me-disse mais recente e irretorquível, quase sempre sussurrado, inalcançável à minha parca compreensão de criança, mas emoldurado pela aura sagrada e misteriosa do interdito\dots 
Um \textit{``oooohhhh!''} seguido de horrorizado silêncio parecia encerrar a conversa. 
Que logo, porém, renascia animada, como fogo sob as cinzas, sempre sublinhada embaixo da mesa pelo entusiasmado balé de pernas e pés. 
Lembro, com nitidez, de uma tarde em que, andando com minha mãe pela rua, ouvi-a chamar a atenção de uma amiga para uma célebre senhora, personagem recorrente nas reuniões à volta da tal mesa, pelo que me lembro, pelas audaciosas e repetidas incursões fora da sagrada cidadela do casamento. 
Olhei na direção indicada e vi uma mulher vestida de preto, grave e majestosa, na penumbra do banco traseiro de um carro escuro e grande.
Divisei-lhe as feições, ainda bonitas na maturidade. 
Séria, ela devolveu o meu olhar fascinado. E o que senti, juro, foi pura admiração.

Dos clãs principais, o da minha mãe era o mais pobre. 
Incluía as irmãs da minha avó Didi, seus maridos e filhos. 
Meu avô João já não tinha mais parentes próximos. 
Dele só conheci um primo engraçado, que vez ou outra aparecia e de novo sumia, como um cometa.  
Embora todas fossem casadas, as Credidio, mulheres invariavelmente fortes e batalhadoras, sobrepunham-se aos machos em autoridade. 
Dessa gente me vieram os impulsos de amar a vida, de superar os revezes sem esmorecer, de correr atrás dos sonhos, mas também a preocupação com as aparências alimentada pelo receio permanente de incorrer em vulgaridade e mau-gosto. 
Alimentavam ambições sociais e acreditavam piamente em cultivar as boas relações para vencer as desvantagens da origem humilde.  
No riso e no pranto, eram pau para toda obra. Com a mesma disposição para encarar bailes e velórios, casamentos, batizados e procissões, envolviam-se em tudo com igual disposição de fazer todo o necessário para que tudo saísse perfeito. 
Davam-se a todas essas empreitadas com sincera devoção e autêntico espírito de caridade e solidariedade. Mas, não dispensavam os holofotes e tinham a deliciosa ingenuidade de nem tentar disfarçar o prazer que sentiam com a repercussão favorável.

O clã do meu pai, os Filpi, embora menos pobre na sua origem, na Itália, já conhecia há muito a inconstância dos fados. 
Lá, na pequena Novi Velia, o patriarca Ângelo viu seus bens esvaírem-se no torvelinho das brigas políticas. 
Aqui, seu filho e meu avô Reginaldo, fazendeiro abastado a quem o café proporcionara razoável fortuna, casa confortável na cidade e educação de qualidade para a prole, perdeu quase tudo na crise de 1929 e teve que levar a família de volta ao recomeço duríssimo na Pedra Branca, única fazenda que lhe restou das quatro ou cinco que chegou a possuir. 
Por conta desses reveses, com certeza, dessa gente me veio um realismo prudente, alicerçado na crença de que estamos cá na terra a serviço e não a passeio, na desconfiança de que todo ídolo oculta pés de barro e de que tudo que reluz quase nunca é ouro, além do mau hábito de trabalhar até o limite das forças. 

O reino das mulheres Filpi era o lar. 
Exímias cozinheiras, incansáveis no lavar, passar, esfregar, bordar e tecer, foram condenadas à vida espartana pelo atraso da minha avó Teresa, uma mente forjada pelo rígido código dos antigos costumes mineiros. 
Mulher era para servir ao homem, pai, irmão ou marido e não para bater pernas na rua e perder-se como as moças pintadas, ataviadas e oferecidas que se viam por aí, querendo diploma, frequentando bailes, usando saias curtas, dirigindo e até fumando! Quando minha mãe, professora formada, frequentadora contumaz dos bailes do clube, exibindo impecáveis unhas esmaltadas e envergando trajes e penteado da moda ingressou na família, não fosse o apoio imediato do galante sogro Reginaldo, certamente seria rechaçada como péssimo exemplo. 
E pelo resto da vida, mesmo após a morte da Vó Teresa, a relação da minha mãe com as cunhadas, foi contraditória: Lectícia era o farol que iluminava para elas os difíceis caminhos da inserção social e da modernidade. 
Mas mamãe, por muitos e muitos anos, padeceu da necessidade neurótica de provar que, apesar das unhas feitas e das roupas da moda, era páreo para elas no comando de um fogão, de um ferro de passar e no brilho das panelas. 
Daí por que preparar um almoço para as cunhadas era um feito precedido por dias de insuportável e cômica tensão, mas que valiam pelo gosto insuperável da vitória, quando o molho e a massa se apresentavam indiscutivelmente no ponto certo. 
Por outro lado, as tias, com meu ouvido treinado em escutar conversa de adultos, ouvia-as muitas vezes lamentar o destino do irmão, coitado, condenado a trabalhar para sustentar os luxos “daquela gastadeira”.

Os Ópice eram oriundos do casamento de duas irmãs da minha avó materna com dois irmãos recém-chegados da Itália e que vieram estabelecer-se em Araraquara: um hábil alfaiate, chamado Bruno e um não menos hábil carpinteiro chamado José. 
Ao que se conta, ambos exibiam como característica principal um certo refinamento de gostos, incomum entre os pobres imigrantes italianos da região e que provavelmente lhes pareceu suficiente para justificar uma postura algo arrogante e prepotente que sempre os distinguiu, tanto quanto os rompantes de temperamento. 
Afora isso, eram muito trabalhadores e competentes. Todos esses traços são ainda bem visíveis na sua descendência. 
O ramo do Tio Bruno, quase todo, enraizou-se em definitivo na cidade e o do Tio José, algum tempo após a morte precoce deste, acabou migrando para a Capital. 
Os filhos do Tio Bruno, que continuaram a viver sob o jugo do pai, jamais amadureceram totalmente e ficaram na minha memória como uma família de gente boa, meio infantilizada e divertida. 
Os órfãos do Tio José, libertos do autoritarismo paterno, com muito mérito e muito trabalho, fizeram carreira e sólida fortuna na capital. 
E foi assim que se transformaram, para toda a gente da minha mãe, na referência suprema do sucesso, do “savoir faire”, o exemplo a ser seguido, fonte de jurisprudência a ser consultada em qualquer circunstância e para qualquer assunto. 
Eram respeitosamente designados como “os de São Paulo” e sempre que vinham em visita aos parentes da província provocavam um alvoroço que, à medida que nós, os mais novos, crescíamos, acabou virando motivo de piadas sempre recebidas como sacrilégio por vovó, mamãe e suas irmãs. 
Já meu pai que, como bom Filpi, nunca foi afeito a idolatrias de qualquer espécie, fechava o tempo e não foram poucas as vezes em que os pobres Ópice acabaram pagando pela admiração incompreensivelmente servil, no entender do marido, que minha mãe devotava às ideias e realizações dos primos.