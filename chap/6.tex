\chapter{}
Fui, ao longo da minha vida escolar, uma boa aluna. 
Não porque fosse uma grande estudiosa, mas sim, porque aprendi a gostar de ler. Meu pai tinha o hábito de, logo depois de retirada a louça do jantar, abrir o jornal sobre a mesa e comentar em voz alta as notícias do dia. 
Parecia tirar um grande prazer disso, de tal forma que comecei a encarapitar-me junto dele, querendo partilhar do ritual, embora fosse ainda muito pequena para entender qualquer coisa. Achando graça, papai começou a soletrar comigo o título do jornal até que ``Folha da Manhã'' foi a primeira coisa que li na vida, com dígrafos, til e tudo. 
Com o passar dos anos, de alguma maneira devo ter percebido que nos livros estavam as respostas para o meu precoce interesse pelas pessoas e pela vida em geral. 
Além disso, logo ficou claro para mim que a leitura proporcionava um agradabilíssimo e rendoso disfarce para a minha dificuldade de memorizar, deficiência preocupante na vigência de uma prática escolar em que a promoção dependia exclusivamente do talento do aluno para reproduzir com absoluta fidelidade o que ensinavam os livros ou o professor. 
Descobri que só guardava o que conseguia entender, digerir e devolver com minhas próprias palavras e a leitura foi a grande responsável pela facilidade com que passei a realizar tais operações. 

Acho que o ponto alto da minha estratégia para ocultar dos professores a incapacidade de decorar aconteceu quando eu cursava a oitava série, no dia em que a mais temida professora do Colégio, D. Cidinha, professora de História, chamou-me para entregar uma prova e perscrutando-me miudamente o rosto, disse:
 
{\itshape``-Teresa, dei-lhe uma nota oito, porque sua narração da revolta pela libertação da Grécia não só está bem redigida, como satisfatoriamente fiel; entretanto, como você omitiu datas e nomes, é verdade que ela se ajustaria à maior parte das revoluções da História. Na pior das hipóteses, você parece ter aprendido bastante sobre a dinâmica dos movimentos revolucionários.''}

Lembrei-me deste fato quando me tornei eu própria uma professora de História e passei a inaugurar meus cursos comunicando aos espantados alunos que naquele ano experimentariam uma novidade: uma professora de História desmemoriada, incapaz de guardar datas e nomes, mas muito interessada nos porquês da História e, principalmente no uso que poderíamos fazer deles para as nossas vidas. 

Mas, ai de mim, nas então pretensiosamente chamadas ``ciências exatas'', com relevo especial para a Matemática, nenhuma estratégia funcionava. 
Tudo me parecia de um autoritarismo monolítico, intolerável, com suas regras, axiomas e princípios. 
Enquanto estudei aritmética saí-me até bem, pois, como lembrou um nosso cronista, acho que Rubem Braga, havia enredo na coisa. 
Havia ética que convidava à solidariedade no esforço de encontrar o troco exato ou de repartir terrenos, laranjas ou bolas de gude de modo que o resultado fosse perfeitamente aceitável para os interessados. 
Quando, porém, ao ingressar no ginásio, fui conduzida por aquele que seria meu único professor da matéria a partir daí, aos tenebrosos domínios da Álgebra, uma parte do meu cérebro entrou em colapso e pareceu soçobrar no oceano escuro da ignorância. 
Para sempre. 
Tornei-me, desde então, incapaz de pensar a vida em números. 
Avessa a medir e contar, fosse lá o que fosse. Mistério considerado insondável por quantos se detiveram a analisar o problema. 

Aquele professor era proclamado um gênio por toda a cidade. 
Muito magro e agitado, avental imaculadamente branco, cheirando a lavanda e cigarro, ele entrava na classe já ditando a lição e escrevendo compulsivamente no quadro negro. Como hieróglifos numa tumba egípcia, intermináveis equações, trinômios, polinômios iam se estampando em ritmo frenético por toda a lousa e, quando esta se esgotava, pelas paredes. 
O giz ia se despedaçando entre os dedos nervosos e espalhando-se pelo chão. 
A intervalos, ele catava os toquinhos e despejava-os numa caixa. 
Tinha uso para eles: atirava-os na cabeça dos distraídos. 
Eu estava entre os seus alvos preferidos. Ao lado da escola, uma imensa paineira lançava ao vento delicados flocos que entravam pela janela da sala e flutuavam como bailarinas minúsculas diante dos meus olhos enlevados.  
Desse devaneio me tirava a voz aguda do professor e o concomitante estalinho do toco de giz sobre a testa:

{\itshape``- {\large\bfseries D o n a T e r e s a F i l p i}, depois não sabe por que não entende   Matemática, não é?''}

Por vias indiretas, devo muito a esse professor. 
Não sei por que, encasquetei com a idéia de encaminhar-me para Medicina quando cheguei ao curso médio ou Colegial, como era chamado então. 
Para tanto, matriculei-me no Científico, opção que enfatizava as Ciências Exatas, com aulas diárias e reforçadas de Matemática. 
Ao dar comigo na classe de primeiro ano, um esgazeado Sr. Ulysses, era assim que ele se chamava, abandonou a sala e dirigindo-se a um telefone ligou para minha mãe exigindo que ela providenciasse a minha imediata transferência para o curso Clássico, que preparava para as faculdades de Línguas e Ciências Humanas. 
Minha mãe obedeceu no ato. Odiei-o por isso e por muito tempo julguei-me frustrada em minha vocação. 
Tornei-me então, educadora. 
Muito tempo depois, haveria de ter um contato mais próximo com a vida de um médico quando, durante a recuperação de uma fratura no pé, trabalhei com meu irmão João na sua clínica de ortopedia. 
Só então é que, no íntimo, agradeci fervorosamente àquele impaciente professor. Jamais teria experimentado pela rotina médica a paixão que me despertou a batalha diária nas salas de aula. 
E mais, a lembrança do martírio que esse professor e eu mutuamente nos infligimos, me orientou a empreender uma revolução no ensino de Matemática do colégio que eu coordenei. 
Jamais consegui aceitar que uma ciência que nasceu para favorecer a harmonia e a ética, educando os homens para a honestidade e para a civilidade, fosse tão desvirtuada por uma pedagogia arbitrária e pretensamente neutra, a pretexto de ser ``científica''. 

A baixa remuneração dos professores, uma triste recorrência na história da educação brasileira, sempre trouxe como efeito, entre tantos outros, a absurda carga horária assumida pela maioria deles e que fazia com que, à semelhança do que me ocorreu com a Matemática, os alunos, numa determinada disciplina, tivessem o mesmo professor por anos a fio. 
Havia deles que ministravam aulas desde o início do primeiro grau até o fim do segundo. 
Quando eram bons, isto era uma sorte incrível na mediocridade reinante. 
Mas, se eram deficientes, o aluno ficava lesado para todo o sempre. 
Aconteceu-me com Geografia, além de Matemática. Ainda hoje ostento uma dificuldade irritante, verdadeiramente limítrofe, para me orientar no espaço. 
Mas, em Português, em compensação, tive, por seis anos consecutivos, um professor maravilhoso que se sustentara na faculdade interpretando radionovelas e que adorava ler para nós. 
Aprendi a língua como quem aprendesse música. 
Nunca soube, como continuo a não saber, regras gramaticais, porque ele não nos aborrecia com isso. 
Aprendi a escrever lendo com ele gente que escrevia bem.  
E ouvindo aquela voz sonora e expressiva realçando o estilo e a elegância do autor. 
Meu velho e querido Professor Jurandyr. Muito aprendi também com Dona Elisa, sobrinha de Mario de Andrade, por curtos anos minha professora de francês e que falava de autores e livros como quem descrevesse delícias culinárias. 
Salivando gulosamente. 
E provocando, em quem a ouvia, uma fome danada de ler. 
E Dona Olga, poetisa de parnasianos versos, uma típica sobrevivente de tempos machadianos, cujo pai tinha sido amigo muito chegado de Ruy Barbosa. 
Anacrônica dama, capaz de exigir dos seus aluninhos, como lição de casa, doze sinônimos para o adjetivo cinzento. 
Que Deus abençoe a todos, quanto lhes devo!
