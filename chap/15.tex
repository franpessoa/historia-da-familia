\chapter{}
Cerca de seis meses depois de chegar a São Paulo, tranquei matrícula no curso de Jornalismo.
Como acontece com a maioria dos jovens, minhas fantasiosas expectativas sobre a profissão demonstraram ser totalmente equivocadas.
Como equivocada também estava minha auto-avaliação relativa à disposição para encarar a rotina de um ``foca'', naquela época.


A pioneira Faculdade de Jornalismo criada pelo jornal ``A Gazeta'' na qual eu fora admitida, era uma incipiente tentativa de dar a essa atividade um cunho mais profissional.
Até então, escrever para jornal era possível para políticos, advogados, literatos que juntassem prestígio ao domínio da língua ou, no caso do repórter, quando acaso, necessidade e oportunidade se juntavam para forjar um.
Jornalistas veteranos e bem conhecidos do público foram postos a nos dar aulas e pareciam perplexos e confusos diante da obrigação de nos transmitirem a ``técnica'' de escrever em jornal, algo sobre o que a maioria deles jamais teve tempo ou a iniciativa de pensar.
Escreviam e pronto.
{\itshape``-- Como é que se ensina vocação ou talento?''}, pareciam se perguntar.
Por esse motivo, as aulas resultavam, quase sempre, em longas digressões sobre as experiências de cada um, ou suas aventuras como correspondentes em lugares exóticos e situações extraordinárias.
Divertido, mas pouco produtivo.
Vendo que a coisa parecia não caminhar para lado algum, alguém da Direção deve ter concluído que o melhor era mesmo apelar para a boa e velha prática.
Assim, logo no princípio do segundo semestre, comunicaram-nos que teria início uma indispensável experiência para nossa adequada formação.
Um a um, fomos designados para um programa de estágio noturno nas delegacias de polícia da cidade de São Paulo, para colher notícias do palpitante submundo da Capital.
 Mesmo agraciada com a gentileza do meu orientador que, levando em conta minha pouca idade e minha condição feminina, destinou-me a uma delegacia mais ou menos central, à beira do Anhangabaú, o efeito sobre mim foi traumático.
Era demais para uma filha de Mathias Filpi, a vida inteira tão protegida do contato com o lado negro da realidade.
Nem cheguei a conhecer o tal antro.
Saí fora.
Decidi dedicar-me de corpo e alma ao meu curso na Sedes e, pior, para alegria da mamãe, substituí a Cásper Líbero pela mais burguesa das escolas, a EAD, ou seja, Escola de Artes e Decoração, uma escola alternativa para patricinhas desocupadas.
Diga-se, a meu favor, que não aguentei muito tempo aquele amontoado de pretensão e futilidade.
Mas, como qualquer saber acaba por ter alguma função, aprendi lá um pouco de História da Arte, pessimamente abordada na faculdade e desenvolvi certa habilidade para harmonizar móveis com tapetes e cortinas.
Mas, principalmente, convenci-me de que ninguém tem o direito de dizer a ninguém como deve morar.
E passei a execrar os decoradores.

O pensionato das Irmãs de São Vicente de Paula, congregação conhecida pelo alvo, engomado e medievalesco arranjo em forma de tricórnio que suas freiras usavam, ficava ao lado da Sedes Sapientiae, numa travessa da Rua Caio Prado.
A casa abrigava cerca de seis dezenas de universitárias, todas vindas do interior, como eu.
Preferi ficar num dos quartos individuais, na realidade celas da antiga clausura, pequenos e desconfortáveis, porém mais caros pelo luxo da privacidade, porque era bem do meu feitio me assegurar do terreno antes de sair fazendo novas relações.
Mas, meu isolamento não duraria muito.
Havia na casa um quarto com três camas, ocupado por apenas duas garotas inconformadas em dividir o ônus pela vaga vazia.
Uma delas, Ana Beatriz, sem nenhum constrangimento, tomou a peito a ingrata tarefa de conquistar minha simpatia e resolver o problema, fazendo de mim a terceira integrante do grupo.
Com seu sotaque limeirense e um sorriso invencível, postou-se à porta que eu mal entreabria à sua chegada, durante uma semana a fio, até convencer-me a acompanhá-la.
Fui apresentada então à sua colega de quarto, uma mineira neurótica, divertida e esguia como um galgo, estudante de Psicologia que à época, ou logo depois, não sei, namorou o Paulo, meu futuro marido.
Juntei-me a elas e aquele quarto, sabe-se lá por que, virou o ponto de encontro predileto da turma.
Todo mundo passava por ali quando voltava das aulas ou aos sábados, domingos e feriados.
Ali se sabia de tudo e de todos e eu reencontrei o prazer de conhecer e estudar a natureza humana.


Naquele tempo, ao chegarem a São Paulo, as ``mocinhas de família'' do interior, pois só mesmo espécimes dessa variedade de juventude feminina aceitariam se submeter às regras disciplinares tão rígidas quanto extemporâneas das Irmãs de São Vicente, experimentavam a um só tempo a emoção de se descobrirem relativamente livres da vigilância familiar e a de tomarem contato com a revolução cultural dos anos 60, cuja bandeira mais vistosa era a da liberação sexual proporcionada pelo advento da pílula.
Cinema Novo, Beatles, Bossa Nova, Existencialismo, Socialismo, Amor Livre, a juventude fervilhava em meio a uma catadupa de novas tendências e idéias, exacerbadas pela agitação política do período que haveria de desembocar no fechamento do regime e nos anos de chumbo da ditadura militar.
E tínhamos, para viver e digerir tudo isso, o tempo que ia das sete da manhã às nove da noite quando, impreterivelmente, nossas inefáveis hospedeiras cerravam as portas do pensionato e apagavam todas as luzes.
Aos sábados, é preciso reconhecer, as irmãs generosamente dilatavam o horário de recolhimento para as vinte e duas horas.
Quem tinha parentes em São Paulo escapava para a casa deles sempre que podia.
Por sorte, pouco tempo antes da minha chegada, Vó Didi, já então viúva, e Tia Maria Angelina tinham se mudado para seu novo apartamento, a duas quadras dali, na Rua Augusta.
Mas, sair do jugo das freiras para cair sob a guarda de avós, tios e primos não acrescentava grande coisa ao nosso projeto de libertação pessoal.
De modo que a participação de boa parte de nós, as mocinhas do pensionato, nos acontecimentos que marcaram a grande revolução dos costumes dos anos sessenta, ficou mesmo em nível de discussão, apenas.

Do mesmo modo, nossa contribuição política aos fatos que antecederam a chamada ``Redentora'', isto é, o golpe militar de 64, assim como aos movimentos de agitação e repressão que o sucederam, foi bastante contida pela vigilância dos parentes e pelos cuidados maternais das freiras, mas, neste caso, não aquelas do pensionato, mas das cônegas agostinianas, que as pobres Irmãs de S.Vicente não tinham alcance para entender o que se passava  além dos muros da clausura.
Já as cônegas agostinianas, uma exceção no universo feminino da Santa Madre Igreja, pois sua formação igualava-se às das mais cultas congregações masculinas, estas tomaram parte ativa nos acontecimentos da época, havendo até algumas delas, como Madre Cristina e Madre Mariângela, respectivamente a diretora do curso de Psicologia e a diretora da minha faculdade de História, que chegaram a merecer a atenção do DOPS, o sinistro aparelho policial da ditadura.
Éramos orientadas e incentivadas por elas a marchar nas passeatas de protesto que mobilizavam a classe estudantil ao longo daqueles anos, mas não deixava de ser ridícula a nossa preparação para tão arriscada empreitada: devíamos ir, sim, mas munidas do arsenal de segurança habitual, lencinhos molhados e ``carvãozinho'', um comprimido negro que trancava os intestinos blindando-os contra os efeitos das chamadas bombas de efeito moral e, o mais importante, vestidas em nossos \textit{tailleurzinhos} à Jackie Kennedy, meias de seda e salto alto (era assim que a maioria de nós frequentava as aulas).
Assim que a polícia atacasse, devíamos entrar nas lojas demonstrando convincente e extasiado interesse pelas vitrines, à feição de modernas Marias Antonietas, indiferentes ao tumulto que nos circundava.
E o engraçado era que funcionava mesmo.
Não era raro que os meganhas nos escoltassem para fora da confusão.

Aconteceu, entretanto, que pouco adiante do Sedes Sapientiae, situava-se a Rua Maria Antonia, campo de batalha onde se digladiavam os alunos do Mackenzie, direitistas fanáticos, acusados de acolher em suas hostes o famigerado CCC (Comando de Caça aos Comunistas) e os mais politizados estudantes da USP: aqueles da Escola de Sociologia e Política, da Faculdade de Economia e da Faculdade de Arquitetura e Urbanismo, ardorosos defensores das correntes de pensamento situadas à esquerda.
As escaramuças iniciais entre uns e outros evoluíram para batalhas verbais, em que palavras de ordem e acusações gritadas ao megafone cruzavam de um lado para outro da rua e chegaram à violência explícita num dia em que coquetéis molotov explodiram, provocando o incêndio do histórico prédio-sede da Universidade de São Paulo.
A repressão baixou com ferocidade e, a partir daí, o governo decidiu apressar a transferência da tradicional instituição pública para o então distante campus do Butantã.
Desalojados, os cabeças do movimento estudantil encontraram na insuspeitada Sedes Sapientiae, aos olhos dos militares uma inofensiva escola ``espera-marido'' conduzida por freirinhas meio subversivas, é verdade, mas, freirinhas, apesar de tudo, o lugar ideal para prosseguir com suas articulações, confiados na inibição que o hábito e a origem aristocrática das nossas monjas imporiam ao governador Abreu Sodré e seus sabujos fardados.
O estratagema funcionou durante algum tempo, o que nos possibilitou a todas um emocionante contato ao vivo e a cores com os lendários e desassombrados líderes do movimento estudantil: Wladimir Palmeira, Paeco, José Serra, José Dirceu (desde então já visto como bonitinho, mas nada confiável) e mais um monte de artistas e intelectuais engajados na luta contra a ditadura.
Lembro-me de um memorável show organizado de improviso no salão nobre da faculdade, a pedido de um hirsuto líder sindical anunciado pelo breve apelido de Lula e que, subindo ao palco, com voz rouca e língua presa, explicou-nos o motivo da iniciativa: arrecadar fundos em benefício dos operários em greve da Nitroquímica.
 Arrebanhadas às pressas pela cidade, uma a uma, foram chegando celebridades como Paulo Autran, Sérgio Ricardo e talentos ainda em cueiros como Chico Buarque, Toquinho e Taiguara.
Lembrança indelével me ficou de uma figura dramaticamente impressionante de menina esculpida em bronze, magra e séria, que cantou uma áspera canção, ``Carcará'', como se incorporasse o pequeno gavião da caatinga, prestes a voar sobre a platéia com seu nariz adunco, os dedos em garra, o olhar duro e brilhante de predador.
Maria Bethânia.


Por várias vezes, nossas salas de aula foram palco de assembleias- relâmpago exaltadas, inflamadas, prementes, da União Estadual dos Estudantes (U.E.E.) e da União Nacional dos Estudantes (U.N.E.).
Numa delas, vencendo a timidez, a inibição interiorana, desfeita numa gelatina trêmula, pedi a palavra e falei, nem me lembro o que, tomada pelo clima emocional, pela convicção imperante, absoluta e contagiosa de que nos cabia salvar o Brasil, e quiçá o mundo, da voracidade do imperialismo norte-americano.
A partir daí criei coragem, saí em campo, fui inscrita como segunda ou terceira secretária numa chapa da U.E.E concorrente a eleições adrede proibidas pelos militares, atirei-me com entusiasmo às passeatas, corri dos policiais na Rua Pinheiros e me refugiei numa padaria onde tive que ouvir desaforos de trabalhadores cansados e exasperados pelo transtorno que os impedia de voltar para casa, enquanto lá fora as bombas explodiam com estardalhaço.
E numa das mais assustadoras experiências deste período louco, vi-me trancada, um dia, com mais três centenas de estudantes no convento dos dominicanos, no célebre cerco que opôs a um bando de frades e àquele punhado de adolescentes, um contingente do II Exército armado até os dentes, capaz de fazer amarelar qualquer liderança do tráfico dos morros cariocas.
Após horas de laboriosas negociações entre Igreja e Estado que avançaram pela tarde e entraram pelo início da noite, nos permitiram sair, afinal, um a um, famintos e assustados, esgueirando-nos através de um corredor polonês de escudos e cassetetes, portando nossos livros e nossas únicas armas - guarda-chuvas, porque chovia até não poder mais naquele dia em São Paulo.

De outra vez, numa das últimas e mais reprimidas passeatas de estudantes, intelectuais e operários paulistas, marchávamos da Praça da República, enveredando pela Barão de Itapetininga em direção ao centro velho de São Paulo, quando os militares nos cercaram, ostentando o melhor do seu aparato bélico, em que se destacavam a cavalaria do exército, a polícia marítima e seus gigantescos porretes, cães, tanques, urutus e brucutus.
Orientados a isolar a cabeceira da multidão, onde supostamente estavam as lideranças do movimento subversivo, os soldados cortaram o desfile ao meio, o que nos permitiu a nós, que caminhávamos na retaguarda, buscar refúgio nos prédios próximos.
Com algumas amigas, pernas bambas de terror, fui parar, sei lá como, no décimo segundo andar de um daqueles edifícios, num minúsculo quarto-e-sala cujo morador, um senhor de idade, piedosamente nos abriu a porta.
 Escapamos por um triz.
Minha futura cunhada, colega de sala e de pensionato, Sila Maria, que estava mais à frente, teve que agachar-se no asfalto, tomou algumas bordoadas e foi metida num camburão.
Mesmo de \textit{tailleur} e salto alto.
Livrou-a da cana e de coisa muito pior, quem sabe, o parentesco com general estrelado.


Passado o perigo, já no início da noite, voltei para casa matutando sobre a conveniência de abandonar minha recém iniciada carreira de militante política.
Uma das regras fundamentais do meu código pessoal de conduta sempre foi a de não me meter \textbf{deliberadamente} em situações cujas consequências eu não pudesse enfrentar, por minha conta e risco.
E, definitivamente, olhando cara a cara a brutalidade e o despropósito da reação do governo aos que se opunham à ditadura recém instalada no país e, por outro lado, testemunhando atitudes emocionais e atabalhoadas de muitos amigos e conhecidos cujas vidas promissoras e mal despontadas os porões da repressão trituraram e devolveram em cacos, ou simplesmente ceifaram sem piedade, decidi que era hora de parar.
Havia pontos obscuros na história toda.
A ditadura era intolerável, isto era certo.
Mas, pelo quê mesmo íamos substituí-la?  Isto já não parecia tão claro.
Ou, pelo menos, algumas das propostas anunciadas sobre os palanques improvisados não me pareciam merecedoras dos riscos suicidas a que nos expúnhamos, fosse pela inexequibilidade face às circunstâncias, fosse pela radicalidade algo suspeita.


No ano seguinte terminei meu curso.
Tinha decidido militar, sim, por um mundo melhor, mas do modo como sabia e podia.
Exercendo minha profissão.
Indo além do mister de ensinar História.
Sendo educadora.
Mas, como tantas vezes comentei com meus alunos, provavelmente não teria chegado a desenvolver uma consciência política tão clara do meu papel na sociedade, se não tivesse feito parte dessa geração que viveu o que seria, por muitas décadas, o canto do cisne da militância política juvenil e idealista neste país.
 Ainda não surgiu outra.
Sem o fio condutor de um ideal, o fantástico potencial de realização dos jovens vem se perdendo em desordem pelo ambiente.
Desperdiçado no vandalismo e na violência gratuita das gangues, consumido pelas drogas, desviado para o consumismo obsessivo, para a rebeldia sem causa, para o culto tão exagerado quanto fútil ao próprio corpo.
É uma pena.
Queira Deus lhes seja restituído tão cedo quanto possível o direito de sonhar utopias, a ventura de repensar o mundo com a onipotência, a ousadia, a vitalidade e a paixão algo irresponsável que só na mocidade nos é permitido viver.

