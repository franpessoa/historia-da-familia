\documentclass[12pt, a5paper, twoside]{book}

\usepackage{float} % incluir imagens
\usepackage[export]{adjustbox} % centralizar
\usepackage{subcaption} % legendas
\usepackage{graphicx}
\graphicspath{ {pic/} }

\usepackage[pass]{geometry}

\usepackage{setspace}
\onehalfspacing{}

\usepackage{indentfirst}
\usepackage[skip=0pt, indent=1.25cm]{parskip}

\usepackage{ragged2e}
\justifying{}

\usepackage[portuguese]{babel}
\usepackage{hyphsubst}

\usepackage{fontspec}

\newenvironment{dedication}{
    \clearpage
    \thispagestyle{empty}
    \vspace*{\stretch{3}}
    \small
    \itshape
    \raggedright
}{\par\vspace*{\stretch{1}}\clearpage\normalfont} % taken from https://tex.stackexchange.com/questions/476646/how-put-title-in-a-dedication

\title{História da Família, Parte II}
\author{Teresa Cristina}
\date{2025}

\renewcommand{\thechapter}{\Roman{chapter}}

\begin{document}

\begin{titlepage}
\maketitle 
\end{titlepage}

\begin{dedication}
Aos meus filhos, para que, tal como ensinou Terêncio, nada que seja humano, seja-lhes estranho. 
Aos que cobrarem maior fidelidade aos fatos, saibam que o que aqui vai contado, vai tal e qual a emoção esculpiu na alma e a memória arquivou sem contestar. 
Pois nunca sabemos com exatidão o que de fato acontece. 
O que fica é o que nos parece e como nos parece no momento do acontecido. 
\end{dedication}

\tableofcontents

\chapter{}
Sou do tempo bom em que os adultos se livravam das crianças sem culpa, mandando: ``Vão brincar lá fora!''.
E ``lá fora'' era o quintal, o maravilhoso país da infância.
Sobretudo quando se vivia numa cidade do interior. 
Era o espaço de descobrir e pensar o mundo. 
Observar as formigas na sua labuta incessante, as lagartixas subindo pelo velho muro meio rachado, as borboletas iridescentes, o rastro prateado e gosmento desenhado pelos vagarosos caracóis.   
Contemplar a incessante migração das nuvens que o vento ora esfiapava, ora ia juntando em formas caprichosas de bichos, árvores, gigantes, castelos, anjos e às vezes mapas, como aqueles pendurados nas paredes da sala de aula. 
Sentir o cheiro da roupa lavada e embebida de sol, pendurada no varal; da fumaça desprendendo-se da lenha queimada no fogão; da terra molhada, da moita de arruda, das rosas, do jasmim, do manacá; do bafio úmido de sepulcro que emanava da escura caverna do porão. 
Deitada de costas nas pedras quentes da calçada, abandonava-me ao sol. O calor invadia aos poucos meu corpo, amolecendo braços e pernas e eu os imaginava penetrando terra adentro, como raízes. 
Imóvel, escutava o zunido das abelhas nas flores, os recados insistentes do ``fogo apagou'' e do bem-te-vi vindos lá das bandas do cemitério, o canto dos sabiás e a algazarra dos pardais e andorinhas sobrevoando a velha paineira do almoxarifado da prefeitura bem ali, do outro lado da rua. 
Na marcenaria, alguns quarteirões acima, o som plangente da serra cortando madeira dividia com o sino da torre da Matriz a tarefa de marcar regularmente a passagem das horas. 
Vez ou outra, roncando em agonia a cada mudança de marcha, os velhos ônibus a querosene passavam sacolejando nos paralelepípedos, rua acima; os carros ainda eram raros. 
E, do outro lado do muro, do barracão da fábrica do meu pai, chegava o som oceânico, abafado e constante das máquinas torcendo os fios de algodão para transformá-los nos novelos de barbante que garantiam o sustento da nossa família e a minha doce vadiagem de criança.

As cores, formas e textura das flores e folhagens me fascinavam. 
Recobriam meus desenhos infantis e emolduravam em cercaduras delirantes as páginas dos meus cadernos de escola.

Na opinião da minha mãe, as flores do nosso quintal deviam obedecer à moda, como tudo o mais. 
Lembro-me da fase dos crisântemos, dálias e crisandálias repolhudas, logo repudiadas em favor das rosas que, por sua vez, acabaram arrancadas para dar lugar às palmas de Santa Rita e às gérberas. 
Essas mudanças eram quase sempre decididas por volta do Dia de Finados, quando as parentes da Capital chegavam sobraçando as novidades compradas nas floriculturas do Arouche. 
Houve uma vez em que a matriarca das hostes paulistanas, Tia Angelina, desceu do trem exibindo buquês de soberbos e inacreditáveis gladíolos azuis. Foi o quanto bastou para que as nossas pobres palmas de Santa Rita caíssem em desgraça. 
``Coisa mais calú!'', decretou minha mãe. 
Era sua expressão favorita para o que lhe parecia vulgar ou caipira. O projeto do novo jardim teve, porém, que ser adiado. 
De tanto mexer na terra, nas sucessivas reformas dos nossos canteiros, mamãe arrumou uma infecção persistente nos dedos que acabou por lhe fazer caírem as unhas. 
Até um médico especialista em São Paulo precisou ser consultado, já que o nosso habitual curandeiro, o Joãozinho da Farmácia, amigo de longa data do papai, não conseguiu resolver a tal eczema.

De qualquer modo, outros e mais amplos planos de mudança começavam a agitar minha família, pois mamãe começara a achar ``calús'' não só as flores, como todo o quintal, a casa antiga de ``parede a meia'' e banheiro fora, os móveis escuros, ``art-déco''. 
Papai prosperava nos negócios e mamãe já podia sonhar mais alto: uma casa moderna, com mais cômodos, armários embutidos, num bairro novo e mais nobre. 
Prenunciando a nova fase, nossa antiga sala de jantar desapareceu para dar lugar a uma mobília nova e aromática, de jacarandá torneado, em que se destacava uma vistosa cristaleira apropriadamente recheada com copos de cristal recém-adquiridos e que tilintavam perigosamente a cada passo nosso sobre o velho assoalho. 
Nossas caminhas \textit{Patente} foram substituídas por outras, artisticamente torneadas também e os velhos colchões de crina foram para o lixo. 
Passamos a dormir em modernos colchões de mola. Um tapeceiro alemão veio tirar as medidas para fabricar sofá, poltrona e cortinas para a sala de estar. 
Coroando todo esse luxo, um dia desembarcou em nosso portão a peça central da nova decoração: um piano de nogueira, novo em folha, que se converteria num dos grandes instrumentos de tortura da minha juventude. 
Porque então, junto com o quintal e a velha casa, ficava para trás também a minha infância. 
Eu me tornara uma adolescente e, em breve, nós mudaríamos para a casa da Av. D. Pedro II, n.º 1273, defronte ao Clube de Campo do Araraquarense. Corria o ano de 1958.

\begin{figure}[H]
\centering
\includegraphics[width=0.5\linewidth]{1/maria-joão.png}
\caption{A família na porta da casa nova. No colo da
Maria Lúcia, o recém-chegado João.}
\end{figure}
\chapter{}
Para o bem e para o mal, minha infância transcorreu numa pequena galáxia de clãs interligados, de origem imigrante e pobre, dos quais os mais importantes eram os Filpi, os Credidio e os Ópice, todos com raízes na Itália. 
Dezenas de olhos e bocas para censurar e orientar e montes de braços para acarinhar e acolher, e uma intensa experiência humana para vivenciar, recolher, degustar e aprender. 

Nossa antiga sala de jantar exibia uma mesa particularmente desgraciosa, quadradona e pesada, mas que, para mim, tinha um encanto especial: embutia um pequeno armário no seu único e largo pé central, onde eram guardados os jornais velhos. 
Um dia descobri que eu cabia direitinho naquele esconderijo e ali, insuspeitada, colhi minhas primeiras e fortes impressões do mundo adulto.  
Era à volta daquela mesa que as visitas se reuniam para o café. 
Era divertido acompanhar, debaixo dela, o movimento das pernas inchadas e nodosas de varizes das tias-avós, na excitação de comentar os acontecimentos da cidade e a vida alheia. 
Quando mamãe e suas irmãs aparteavam a discussão, vez ou outra, contrapondo argumentos mais modernos, arejados por visões do pós-guerra de liberação feminina, difundidas pelo cinema norte-americano quase sempre, as vozes se alteravam, os pés se agitavam e, na ânsia de vencer a contenda, de algum ponto da mesa vinha a cartada definitiva, o disse-me-disse mais recente e irretorquível, quase sempre sussurrado, inalcançável à minha parca compreensão de criança, mas emoldurado pela aura sagrada e misteriosa do interdito\dots 
Um “oooohhhh!” seguido de horrorizado silêncio parecia encerrar a conversa. 
Que logo, porém, renascia animada, como fogo sob as cinzas, sempre sublinhada embaixo da mesa pelo entusiasmado balé de pernas e pés. 
Lembro, com nitidez, de uma tarde em que, andando com minha mãe pela rua, ouvi-a chamar a atenção de uma amiga para uma célebre senhora, personagem recorrente nas reuniões à volta da tal mesa, pelo que me lembro, pelas audaciosas e repetidas incursões fora da sagrada cidadela do casamento. 
Olhei na direção indicada e vi uma mulher vestida de preto, grave e majestosa, na penumbra do banco traseiro de um carro escuro e grande.
Divisei-lhe as feições, ainda bonitas na maturidade. 
Séria, ela devolveu o meu olhar fascinado. E o que senti, juro, foi pura admiração.

Dos clãs principais, o da minha mãe era o mais pobre. 
Incluía as irmãs da minha avó Didi, seus maridos e filhos. 
Meu avô João já não tinha mais parentes próximos. 
Dele só conheci um primo engraçado, que vez ou outra aparecia e de novo sumia, como um cometa.  
Embora todas fossem casadas, as Credidio, mulheres invariavelmente fortes e batalhadoras, sobrepunham-se aos machos em autoridade. 
Dessa gente me vieram os impulsos de amar a vida, de superar os revezes sem esmorecer, de correr atrás dos sonhos, mas também a preocupação com as aparências alimentada pelo receio permanente de incorrer em vulgaridade e mau-gosto. 
Alimentavam ambições sociais e acreditavam piamente em cultivar as boas relações para vencer as desvantagens da origem humilde.  
No riso e no pranto, eram pau para toda obra. Com a mesma disposição para encarar bailes e velórios, casamentos, batizados e procissões, envolviam-se em tudo com igual disposição de fazer todo o necessário para que tudo saísse perfeito. 
Davam-se a todas essas empreitadas com sincera devoção e autêntico espírito de caridade e solidariedade. Mas, não dispensavam os holofotes e tinham a deliciosa ingenuidade de nem tentar disfarçar o prazer que sentiam com a repercussão favorável.

O clã do meu pai, os Filpi, embora menos pobre na sua origem, na Itália, já conhecia há muito a inconstância dos fados. 
Lá, na pequena Novi Velia, o patriarca Ângelo viu seus bens esvaírem-se no torvelinho das brigas políticas. 
Aqui, seu filho e meu avô Reginaldo, fazendeiro abastado a quem o café proporcionara razoável fortuna, casa confortável na cidade e educação de qualidade para a prole, perdeu quase tudo na crise de 1929 e teve que levar a família de volta ao recomeço duríssimo na Pedra Branca, única fazenda que lhe restou das quatro ou cinco que chegou a possuir. 
Por conta desses reveses, com certeza, dessa gente me veio um realismo prudente, alicerçado na crença de que estamos cá na terra a serviço e não a passeio, na desconfiança de que todo ídolo oculta pés de barro e de que tudo que reluz quase nunca é ouro, além do mau hábito de trabalhar até o limite das forças. 

O reino das mulheres Filpi era o lar. 
Exímias cozinheiras, incansáveis no lavar, passar, esfregar, bordar e tecer, foram condenadas à vida espartana pelo atraso da minha avó Teresa, uma mente forjada pelo rígido código dos antigos costumes mineiros. 
Mulher era para servir ao homem, pai, irmão ou marido e não para bater pernas na rua e perder-se como as moças pintadas, ataviadas e oferecidas que se viam por aí, querendo diploma, frequentando bailes, usando saias curtas, dirigindo e até fumando! Quando minha mãe, professora formada, frequentadora contumaz dos bailes do clube, exibindo impecáveis unhas esmaltadas e envergando trajes e penteado da moda ingressou na família, não fosse o apoio imediato do galante sogro Reginaldo, certamente seria rechaçada como péssimo exemplo. 
E pelo resto da vida, mesmo após a morte da Vó Teresa, a relação da minha mãe com as cunhadas, foi contraditória: Lectícia era o farol que iluminava para elas os difíceis caminhos da inserção social e da modernidade. 
Mas mamãe, por muitos e muitos anos, padeceu da necessidade neurótica de provar que, apesar das unhas feitas e das roupas da moda, era páreo para elas no comando de um fogão, de um ferro de passar e no brilho das panelas. 
Daí por que preparar um almoço para as cunhadas era um feito precedido por dias de insuportável e cômica tensão, mas que valiam pelo gosto insuperável da vitória, quando o molho e a massa se apresentavam indiscutivelmente no ponto certo. 
Por outro lado, as tias, com meu ouvido treinado em escutar conversa de adultos, ouvia-as muitas vezes lamentar o destino do irmão, coitado, condenado a trabalhar para sustentar os luxos “daquela gastadeira”.

Os Ópice eram oriundos do casamento de duas irmãs da minha avó materna com dois irmãos recém-chegados da Itália e que vieram estabelecer-se em Araraquara: um hábil alfaiate, chamado Bruno e um não menos hábil carpinteiro chamado José. 
Ao que se conta, ambos exibiam como característica principal um certo refinamento de gostos, incomum entre os pobres imigrantes italianos da região e que provavelmente lhes pareceu suficiente para justificar uma postura algo arrogante e prepotente que sempre os distinguiu, tanto quanto os rompantes de temperamento. 
Afora isso, eram muito trabalhadores e competentes. Todos esses traços são ainda bem visíveis na sua descendência. 
O ramo do Tio Bruno, quase todo, enraizou-se em definitivo na cidade e o do Tio José, algum tempo após a morte precoce deste, acabou migrando para a Capital. 
Os filhos do Tio Bruno, que continuaram a viver sob o jugo do pai, jamais amadureceram totalmente e ficaram na minha memória como uma família de gente boa, meio infantilizada e divertida. 
Os órfãos do Tio José, libertos do autoritarismo paterno, com muito mérito e muito trabalho, fizeram carreira e sólida fortuna na capital. 
E foi assim que se transformaram, para toda a gente da minha mãe, na referência suprema do sucesso, do “savoir faire”, o exemplo a ser seguido, fonte de jurisprudência a ser consultada em qualquer circunstância e para qualquer assunto. 
Eram respeitosamente designados como “os de São Paulo” e sempre que vinham em visita aos parentes da província provocavam um alvoroço que, à medida que nós, os mais novos, crescíamos, acabou virando motivo de piadas sempre recebidas como sacrilégio por vovó, mamãe e suas irmãs. 
Já meu pai que, como bom Filpi, nunca foi afeito a idolatrias de qualquer espécie, fechava o tempo e não foram poucas as vezes em que os pobres Ópice acabaram pagando pela admiração incompreensivelmente servil, no entender do marido, que minha mãe devotava às idéias e realizações dos primos.
\chapter{}
Em meados do século XX, o castigo físico para a meninada não só continuava em pleno vigor, como em doses razoáveis era considerado instrumento eficiente na construção do caráter infantil. 
Já não se usava a palmatória nas escolas, mas ``a régua cantava'' regularmente nas salas de aula sem que alunos ou pais pensassem em recorrer à justiça por causa disso. 
Era até de bom-tom, como testemunho de confiança e parceria, que estes últimos, ao entregar seus filhos aos mestres, delegassem a eles a prerrogativa de puxar-lhes as orelhas sempre que necessário.

Herança do meu avô Reginaldo, na minha casa havia um \textit{rabo-de-tatu}, uma espécie de chicote feito com pouco mais de três palmos de couro de sola, grosso, arrematado por um cabo trançado em torno de uma argola de metal pela qual ficava pendurado atrás da porta da cozinha. 
Originalmente, tal instrumento era usado para espicaçar a montaria. Como meu avô era exímio cavaleiro e vivia de chicote em punho, deve ter sido só uma questão de oportunidade para que o artefato se convertesse em recurso pedagógico. 
Por sorte, na maioria das vezes, o rabo de tatu da minha casa servia mais à intimidação do que ao uso propriamente dito. 
As surras ``para valer'' eram ministradas sempre pelo meu pai que, explosivo como era, preferia usar suas temíveis manoplas ao invés de perder tempo indo atrás do tal relho. 
Minha mãe é que às vezes o usava para potencializar a escassa força dos seus punhos femininos. Mas, na maior parte do tempo ela recorria apenas aos beliscões, aos tapas e às ameaças:

{\small\itshape “ -- Deixa estar! Vou contar tudo ao seu pai quando ele chegar e aí eu quero ver!”}

Essa história de apanhar me parecia revoltante pelo que atentava contra a dignidade das pessoas. 
Sentia-o mesmo quando muito menina para sabê-lo. 
Minha mãe apenas me irritava quando me batia, mas com meu pai era diferente.  
Principalmente porque eu tinha uma forma peculiar de reagir ao medo e que redundava numa humilhação ainda maior: eu me urinava todinha, só de pressentir a iminência do seu ataque de cólera. 
Bastava que ele entrasse porta adentro me chamando e escandindo as sílabas do meu nome, anúncio inequívoco das bofetadas que viriam. 
As pernas amoleciam e eu me agachava, passiva, em meio à poça que se alastrava em torno de mim. Com o tempo, percebi que o meu terror parecia aplacar-lhe a ira. 
Eu acabava apanhando menos que meu pobre irmão que, rebelde, usava toda a agilidade das suas magras perninhas para fugir, certo de que papai, pesado e lento, jamais conseguiria alcançá-lo. 
Mas a estratégia se revelava pior. Mais enfurecido, quando lhe punha as mãos, meu pai parecia perder a noção da absurda desproporção entre o seu tamanho e o da frágil criança. 
Mais de uma vez circunstantes assustados intervieram com medo de que ele arrebentasse o menino. 

Apanhamos com alguma constância, meu irmão e eu até a adolescência. 
Não porque evidenciássemos uma inclinação acentuada para a delinqüência. 
Ao contrário. 
Éramos bons alunos e nosso repertório de travessuras nunca excedeu o limite do esperado numa época em que as crianças “eram para ser vistas, mas não ouvidas”. 
Reginaldo era um garoto vivo e muito esperto. 
Foi sempre festejado por essas características e, até onde me lembro, meus pais pareciam orgulhosos em ressaltá-las sempre que se referiam ao filho. 
Apanhar por causa disso devia causar uma grande confusão na cabeça dele. Já eu, que Vó Didi chamava de “minha pata-choca”, precisava de uma razão muito forte para sair da minha habitual bonomia. 
Também não conseguia entender o porquê das surras.

\begin{figure}[H]
\centering
\includegraphics[width = 0.9\linewidth]{2/Reginaldo.png}
\caption{Reginaldo e eu por volta de 1950.}
\end{figure}

Uma das primeiras grandes sovas que levei, lembro, foi no Jardim da Infância porque havia, na escola, a atribuição semanal de uma “Medalha de Disciplina” aos que demonstravam melhor comportamento ao longo da semana e num sábado, excepcionalmente, eu não a trouxe para casa. 
Minha professora me deixara tomando conta dos meus vinte e poucos coleguinhas para dar uma escapadela, sabe-se lá para onde e um deles me pediu que deixasse a turma escrever na lousa. 
Com o discernimento a mim conferido pelos meus seis anos, achei que aquela seria uma boa forma de mantê-los ocupados. 
Não foi. 
Quando a professora retornou, havia giz para todo lado e um realizado e barulhento grupo de pirralhos tinha transformado não só a lousa como todas as paredes da sala numa exposição surrealista, sem que eu, impotente, conseguisse contê-los. 
Meu pai nem quis ouvir a história. 
Ainda hoje acho que a professora, D. Lídia, é que devia ter apanhado em meu lugar. 
 
Na família da minha mãe esse tratamento não existia. 
A menos que se levasse em consideração os beliscões e alfinetadas que marcavam os momentos relevantes dos sermões da Vó Didi às filhas adolescentes, enquanto lhes experimentava as costuras. 
Mamãe achava que ela reservava esse momento de propósito para os corretivos. O
Vô João, ao contrário, não só jamais se agastava como se apressava em protegê-las das tempestades maternas. 

Já na família do meu pai, o emprego da força física para dirimir questões de qualquer natureza era um valor muito apreciado. 
Vó Teresa, contava minha mãe, jactava-se de que seus filhos eram homens de resolver qualquer parada “com um soco só”. 

Um fato lembrado com divertida alacridade pelos irmãos Filpi, e exemplo amplamente divulgado do que acabo de relatar, dizia respeito a uma façanha perpetrada por um deles, o Tio Bepe, um rotundo peso-pesado de cento e trinta quilos, quando desconfiou que Tia Antonieta e Tia Glória, suas irmãs e noivas de dois irmãos, proprietários de uma vidraçaria, estavam sendo iludidas. 
Como a data do casamento ia sendo prorrogada ``ad infinitum'' pela família dos rapazes, ele decidiu tomar satisfação com os infelizes vidraceiros que, pálidos de terror, mal conseguiram entender o motivo da intempestiva visita, comunicado aos berros por cima do crepitar dos vidros pulverizados a murros por aquele búfalo ensandecido. 
Nem é preciso acrescentar que, com o estabelecimento, veio abaixo também o sonho das pobres moças de um dia se casarem. 

O castigo físico como recurso educativo era praticado com naturalidade e frequência entre os Filpi. 
Ouvi minhas tias aludirem, mais de uma vez, a uma bizarra norma pela qual, em crianças, quando um cometia uma falta passível de castigo, todos apanhavam para que um não caçoasse do outro. 
E papai lembrava com muito ressentimento uma passagem em que, muito pequeno ainda, distraíra-se observando formigas, deitado de bruços, espichado ao sol, ao invés de varrer o terreiro de café como lhe tinha sido ordenado.  
Sem dizer água vai, o avô Reginaldo, lá da varanda, fez estalar o comprido látego de conduzir charrete e desenhou-lhe um vergão em brasas ao comprido das costas. 
Revoltou-o demais o gesto tão violento quanto traiçoeiro. 
O que não o impediu de, ao crescer, acrescentar sua própria contribuição ao longo prontuário de brutalidades dos Filpi. 
Vovó Teresa nada ficava a dever ao marido no que diz respeito ao emprego da força como recurso disciplinar. 
Ao contrário, dado que Vovô se ausentava com freqüência, na maioria das vezes era a ela que cabia exemplar a numerosa prole. 
Mamãe contava ter presenciado uma cena em que tia Antonieta, a que mais ansiava por uma história de vida diferente daquela que levava em regime de semi-clausura, foi pedir à mãe permissão para continuar os estudos. 
Em resposta, recebeu tamanho safanão que perdeu o fôlego e a fala por longos e angustiantes segundos. 
E ela já era uma mocinha, a coitada.

Não é de admirar que os irmãos Filpi, incluídas aí as mulheres, uma vez adultos, fossem estimados e respeitados pelas numerosas qualidades de diligência, honestidade, generosidade, disponibilidade para com as necessidades do outro, mas igualmente temidos pela fúria demolidora que lhes podia brotar das entranhas quando enraivecidos.  
Naquele tempo, porém, no que toca aos machos da família, esta característica não chegava a causar grande escândalo porque encontrava certo amparo na mentalidade que predominava, como ainda predomina em alguns rincões da nossa sociedade, de que só pelo uso da força bruta um homem se legitima como tal. 
E o público feminino, não se pode negar, só recentemente começou a limitar seu entusiasmo por esse gênero de exibição. 
Mais de uma vez surpreendi uma expressão de orgulho na fisionomia da minha mãe quando meu pai se impunha pela força dos seus músculos. 
E ele nem precisava disso. 
Tinha estatura moral mais do que suficiente para conquistar naturalmente a admiração e o respeito de quantos o conheceram, inclusive e principalmente nossos, dos seus filhos. 

De qualquer maneira, no caso do meu pai, pelo menos, tenho certeza de que a necessidade de afirmar sua autoridade por esse meio era uma das injunções que o atormentavam. 
Tanto que, após cada um desses surtos de raiva primitiva e indomável, ele parecia consumir-se em arrependimento e buscava por todos os modos uma maneira de compensar a vítima: um presente, um passeio, um agrado qualquer. 
Mas jamais conseguiu reagir de outra maneira a uma situação em que se sentisse desafiado.

\begin{figure}[H]
\centering
\includegraphics[width = 0.5\linewidth]{2/Familia.png}
\caption{Nossa família, no início dos anos 50.}
\end{figure}

Todos sabem que é contra o muro sólido do poder adulto que os jovens afiam seus dentes e garras. 
Seria preciso que meu pai fosse uma personalidade emocionalmente mais segura para suportar as investidas ditadas pela nossa imaturidade. 
Assombrava-o a possibilidade de cometer erros ou perder as rédeas na condução da família e na educação dos filhos. 
Porque, em nosso meio, esse seria o pior dos fracassos, o mais imperdoável, social ou pessoalmente falando. 
Ao contrário do que se ouve muitas vezes hoje, era opinião corrente que um filho desencaminhado expunha ao mundo a incompetência dos pais. 
Então, naquele tempo, mais que hoje em dia, à medida que a gente evoluía previsivelmente na direção da independência, os problemas em casa tendiam a aumentar na mesma proporção e o chamado “choque de gerações” sobrevinha com impacto terrível sobre as relações familiares. 
De alguma maneira, traumas a parte, esse conflito acabava por exercer uma força centrífuga de tal monta que raras vezes um jovem, depois de experimentá-la, desejava voltar para casa, a não ser depois que achasse um rumo na vida, ou seja, um diploma, um trabalho e até um casamento meio engatilhado. 
Essa podia não ser a forma mais adequada de pais promoverem a autonomia dos filhos, mas funcionava, sem sombra de dúvida. 
Os raros homens ou mulheres que permaneciam em casa, na dependência dos pais, além dos vinte e poucos anos, eram considerados indivíduos frustrados no seu desenvolvimento e olhados com um misto de piedade e estranheza.
As pessoas se perguntavam o que podia ter dado errado.  

Pode-se entender, neste contexto, a insegurança e as reações do meu pai, tendo sido ele próprio, além de tudo, um adolescente um tanto desajustado, sensível e difícil, que por pouco “não se perdeu por aí”, segundo contavam seus irmãos. 
Nele, como me acostumei a dizer, a rebeldia corcoveou até o fim sob a sela pesada da responsabilidade de homem de negócios e chefe de família. 
A repressão violenta exercida sobre nós, sobretudo os seus filhos mais velhos, hoje eu acredito, era também resultado de um esforço hercúleo para exorcizar sua própria inconformidade.

Aos catorze anos decidi que bastava. 
Quando num acesso de fúria meu pai levantou a mão para mim, encarei-o decidida a me defender, fosse como fosse. 
Acho que ele percebeu. Nunca mais repetiu o gesto. 
Paradoxalmente, entretanto, aquela manzorra pesada que tanto me apavorava quando se precipitava como aríete na minha direção, ficou na minha lembrança como a mesma que me comunicava tanta confiança e proteção quando se fechava, quente e firme, sobre minha pequena mão de criança.
\end{document}